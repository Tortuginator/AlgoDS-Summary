%%% File encoding is ISO-8859-1 (also known as Latin-1)
%%% You can use special characters just like �,� and �

% Input encoding is 'latin1' (Latin 1 - also known as ISO-8859-1)
% CTAN: http://www.ctan.org/pkg/inputenc
% 
% A newer package is available - you may look into:
% \usepackage[x-iso-8859-1]{inputenc}
% CTAN: http://www.ctan.org/pkg/inputenx
\usepackage[latin1]{inputenc}

% Font Encoding is 'T1' -- important for special characters such as Umlaute � or � and special characters like � (enje)
% CTAN: http://www.ctan.org/pkg/fontenc
\usepackage[T1]{fontenc}

% Language support for 'english' (alternative 'ngerman' or 'french' for example)
% CTAN: http://www.ctan.org/pkg/babel
\usepackage[english]{babel} 

% Doing calculations with LaTeX units -- needed for the vertical line in the footer
% CTAN: http://www.ctan.org/pkg/calc
\usepackage{calc}

% Extended graphics support 
% There is also a package named 'graphics' - watch out!
% CTAN: http://www.ctan.org/pkg/graphicx
\usepackage{graphicx}

% Extendes support for floating objects (tables, figures), adds the [H] placing option (\begin{figure}[H]) which palces it "Here" (without any doubt).
% CTAN: http://www.ctan.org/pkg/float
\usepackage{float}

% Extended color support
% I use the command \definecolor for example. 
% Option 'Table': Load the colortbl package, in order to use the tools for coloring rows, columns, and cells within tables.
% CTAN: http://www.ctan.org/pkg/xcolor
\usepackage[table]{xcolor} 

% Nice tables
% CTAN: http://www.ctan.org/pkg/booktabs
\usepackage{booktabs}

% Better support for ragged left and right. Provides the commands \RaggedRight and \RaggedLeft. 
% Standard LaTeX commands are \raggedright and \raggedleft
% http://www.ctan.org/pkg/ragged2e
\usepackage{ragged2e}

% Create function plots directly in LaTeX
% CTAN: http://www.ctan.org/pkg/pgfplots
\usepackage{pgfplots}
\pgfplotsset{compat=1.11}
\usepackage{xcolor}
\usepackage{amsfonts}
\usepackage{fdsymbol}
\usepackage{graphicx}
\usepackage[breakable, theorems, skins]{tcolorbox}
\usepackage{scrextend}
\usepackage{multicol}
\usepackage{booktabs}
\usepackage{array}
\usepackage{tabularx}